\documentclass[journal,12pt,twocolumn]{IEEEtran}
%
\usepackage{setspace}
\usepackage{gensymb}
\usepackage{xcolor}
\usepackage{caption}
%\usepackage{subcaption}
%\doublespacing
\singlespacing

%\usepackage{graphicx}
%\usepackage{amssymb}
%\usepackage{relsize}
\usepackage[cmex10]{amsmath}
\usepackage{mathtools}
%\usepackage{amsthm}
%\interdisplaylinepenalty=2500
%\savesymbol{iint}
%\usepackage{txfonts}
%\restoresymbol{TXF}{iint}
%\usepackage{wasysym}
\usepackage{hyperref}
\usepackage{amsthm}
\usepackage{mathrsfs}
\usepackage{txfonts}
\usepackage{stfloats}
\usepackage{cite}
\usepackage{cases}
\usepackage{subfig}
%\usepackage{xtab}
\usepackage{longtable}
\usepackage{multirow}
%\usepackage{algorithm}
%\usepackage{algpseudocode}
%\usepackage{enumerate}
\usepackage{enumitem}
\usepackage{mathtools}
%\usepackage{iithtlc}
%\usepackage[framemethod=tikz]{mdframed}
\usepackage{listings}
\let\vec\mathbf


%\usepackage{stmaryrd}


%\usepackage{wasysym}
%\newcounter{MYtempeqncnt}
\DeclareMathOperator*{\Res}{Res}
%\renewcommand{\baselinestretch}{2}
\renewcommand\thesection{\arabic{section}}
\renewcommand\thesubsection{\thesection.\arabic{subsection}}
\renewcommand\thesubsubsection{\thesubsection.\arabic{subsubsection}}

\renewcommand\thesectiondis{\arabic{section}}
\renewcommand\thesubsectiondis{\thesectiondis.\arabic{subsection}}
\renewcommand\thesubsubsectiondis{\thesubsectiondis.\arabic{subsubsection}}

%\renewcommand{\labelenumi}{\textbf{\theenumi}}
%\renewcommand{\theenumi}{P.\arabic{enumi}}

% correct bad hyphenation here
\hyphenation{op-tical net-works semi-conduc-tor}

\lstset{
language=Python,
frame=single, 
breaklines=true,
columns=fullflexible
}



\begin{document}
%

\theoremstyle{definition}
\newtheorem{theorem}{Theorem}[section]
\newtheorem{problem}{Problem}
\newtheorem{proposition}{Proposition}[section]
\newtheorem{lemma}{Lemma}[section]
\newtheorem{corollary}[theorem]{Corollary}
\newtheorem{example}{Example}[section]
\newtheorem{definition}{Definition}[section]
%\newtheorem{algorithm}{Algorithm}[section]
%\newtheorem{cor}{Corollary}
\newcommand{\BEQA}{\begin{eqnarray}}
\newcommand{\EEQA}{\end{eqnarray}}
\newcommand{\define}{\stackrel{\triangle}{=}}
\newcommand{\myvec}[1]{\ensuremath{\begin{pmatrix}#1\end{pmatrix}}}
\newcommand{\mydet}[1]{\ensuremath{\begin{vmatrix}#1\end{vmatrix}}}
\bibliographystyle{IEEEtran}
%\bibliographystyle{ieeetr}
\providecommand{\nCr}[2]{\,^{#1}C_{#2}} % nCr
\providecommand{\nPr}[2]{\,^{#1}P_{#2}} % nPr
\providecommand{\mbf}{\mathbf}
\providecommand{\pr}[1]{\ensuremath{\Pr\left(#1\right)}}
\providecommand{\qfunc}[1]{\ensuremath{Q\left(#1\right)}}
\providecommand{\sbrak}[1]{\ensuremath{{}\left[#1\right]}}
\providecommand{\lsbrak}[1]{\ensuremath{{}\left[#1\right.}}
\providecommand{\rsbrak}[1]{\ensuremath{{}\left.#1\right]}}
\providecommand{\brak}[1]{\ensuremath{\left(#1\right)}}
\providecommand{\lbrak}[1]{\ensuremath{\left(#1\right.}}
\providecommand{\rbrak}[1]{\ensuremath{\left.#1\right)}}
\providecommand{\cbrak}[1]{\ensuremath{\left\{#1\right\}}}
\providecommand{\lcbrak}[1]{\ensuremath{\left\{#1\right.}}
\providecommand{\rcbrak}[1]{\ensuremath{\left.#1\right\}}}
\theoremstyle{remark}
\newtheorem{rem}{Remark}
\newcommand{\sgn}{\mathop{\mathrm{sgn}}}
\providecommand{\abs}[1]{\left\vert#1\right\vert}
\providecommand{\res}[1]{\Res\displaylimits_{#1}} 
\providecommand{\norm}[1]{\lVert#1\rVert}
\providecommand{\mtx}[1]{\mathbf{#1}}
\providecommand{\mean}[1]{E\left[ #1 \right]}
\providecommand{\fourier}{\overset{\mathcal{F}}{ \rightleftharpoons}}
\providecommand{\ztrans}{\overset{\mathcal{Z}}{ \rightleftharpoons}}
%\providecommand{\hilbert}{\overset{\mathcal{H}}{ \rightleftharpoons}}
\providecommand{\system}{\overset{\mathcal{H}}{ \longleftrightarrow}}
	%\newcommand{\solution}[2]{\textbf{Solution:}{#1}}
\newcommand{\solution}{\noindent \textbf{Solution: }}
\providecommand{\dec}[2]{\ensuremath{\overset{#1}{\underset{#2}{\gtrless}}}}
\numberwithin{equation}{section}
%\numberwithin{equation}{subsection}
%\numberwithin{problem}{subsection}
%\numberwithin{definition}{subsection}
\makeatletter
\@addtoreset{figure}{problem}
\makeatother
\let\StandardTheFigure\thefigure
%\renewcommand{\thefigure}{\theproblem.\arabic{figure}}
\renewcommand{\thefigure}{\theproblem}
%\numberwithin{figure}{subsection}
\def\putbox#1#2#3{\makebox[0in][l]{\makebox[#1][l]{}\raisebox{\baselineskip}[0in][0in]{\raisebox{#2}[0in][0in]{#3}}}}
     \def\rightbox#1{\makebox[0in][r]{#1}}
     \def\centbox#1{\makebox[0in]{#1}}
     \def\topbox#1{\raisebox{-\baselineskip}[0in][0in]{#1}}
     \def\midbox#1{\raisebox{-0.5\baselineskip}[0in][0in]{#1}}
\vspace{3cm}
\title{ 
%\logo{
%}
Pingala Series
%	\logo{Octave for Math Computing }
}
\author{ Siri Chandra Valasa$^{*}$ %<-this  stops a space
}
\maketitle
%\newpage
\tableofcontents
%\renewcommand{\thefigure}{\thesection.\theenumi}
%\renewcommand{\thetable}{\thesection.\theenumi}
\renewcommand{\thefigure}{\theenumi}
\renewcommand{\thetable}{\theenumi}
%\renewcommand{\theequation}{\thesection}
\bigskip
\begin{abstract}
This manual provides a simple introduction to Transforms
\end{abstract}
\section{JEE 2019}
Let 
\begin{align}
	a_n &= \frac{\alpha^{n}-\beta^{n}}{\alpha - \beta}, \quad n \ge 1
	\\
	b_n &= a_{n-1} + a_{n+1}, \quad n \ge 2, \quad b_1 =1
	\label{eq:10-orig-diff}
\end{align}
Verify the following using a python code.
\begin{enumerate}[label=\thesection.\arabic*
,ref=\thesection.\theenumi]
\item 
\begin{align}
	\sum_{k=1}^{n}a_k = a_{n+2}-1, \quad n \ge 1
\end{align}
 \item 
\begin{align}
	\sum_{k=1}^{\infty}\frac{a_k}{10^k} =\frac{10}{89}
\end{align}
 \item 
\begin{align}
	b_n =\alpha^n + \beta^n, \quad n \ge 1
\end{align}
 \item 
\begin{align}
	\sum_{k=1}^{\infty}\frac{b_k}{10^k} =\frac{8}{89}
\end{align}
\solution
Verified 1.1,1.2,1.3,1.4 in the below mentioned code
\begin{lstlisting}
$ https://github.com/sirichandra003/EE3900/blob/master/Assignments/Pingala%20Series/codes/q1.py
\end{lstlisting}
\end{enumerate}
\section{Pingala Series}
\begin{enumerate}[label=\thesection.\arabic*,ref=\thesection.\theenumi]
\item The {\em one sided} $Z$-transform of $x(n)$ is defined as 
%\cite{proakis_dsp}
\begin{align}
	X^{+}(z) = \sum_{n = 0}^{\infty}x(n)z^{-n}, \quad z \in \mathbb{C}
\label{eq:one-Z}
\end{align}
	\item The {\em Pingala} series is generated using the difference equation 
\begin{align}
	x(n+2) = x\brak{n+1} + x\brak{n},  \quad x(0) = x(1) = 1, n \ge 0
	\label{eq:10-pingala}
\end{align}
Generate a stem plot for $x(n)$.
\\
\solution
\begin{lstlisting}
$ https://github.com/sirichandra003/EE3900/blob/master/Assignments/Pingala%20Series/codes/q2.2.py
\end{lstlisting}
\begin{figure}[!htp]
    \includegraphics[width=\columnwidth]{figs/2.2.png}
    \caption{Plot of $x(n)$}
    \label{fig:xn}
\end{figure}
\item Find $X^{+}(z)$.
\\
\solution Taking the one-sided $Z$-transform on both sides of \eqref{eq:10-pingala},
\begin{align}
    &\mathcal{Z}^+\sbrak{x(n + 2)} = \mathcal{Z}^+\sbrak{x(n + 1)} + \mathcal{Z}^+\sbrak{x(n)} \\
    &\sum_{n = 0}^{\infty}x(n+2)z^{-n}=\sum_{n = 0}^{\infty}x(n+1)z^{-n} + \sum_{n = 0}^{\infty}x(n)z^{-n} \\
    &z^2X^+(z) - z^2x(0) - zx(1) = zX^+(z) - zx(0) + X^+(z) \\
    &\brak{z^2 - z - 1}X^+(z) = z^2 \\
    &X^+(z) = \frac{1}{1 - z^{-1} - z^{-2}} \\
    &X^+(z)= \frac{1}{\brak{1 - \alpha z^{-1}}\brak{1 - \beta z^{-1}}}, \quad |z| > \alpha  
    \\ &X^+(z)= \frac{z^2}{\brak{z - \alpha }\brak{z - \beta }}, \quad |z| > \alpha
    \label{eq:X-z}
\end{align}
\item Find $x(n)$. \\
\solution Expanding $X^+(z)$ in \eqref{eq:X-z} using partial fractions, we get
\begin{align}
    X^+(z) &= \frac{1}{\brak{\alpha - \beta}z^{-1}}\sbrak{\frac{1}{1 - \alpha z^{-1}} - \frac{1}{1 - \beta z^{-1}}} \\
           &= \frac{1}{\brak{\alpha - \beta}}\sum_{n = 0}^{\infty}\brak{\alpha^n - \beta^n}z^{-n + 1} \\
           &= \sum_{n = 1}^{\infty}\frac{\alpha^{n} - \beta^{n}}{\alpha - \beta}z^{-n + 1} 
\end{align}
Substitute $n :=n+1$
\begin{align}
    x(n) = \frac{\alpha^{n + 1} - \beta^{n + 1}}{\alpha - \beta}u(n) = a_{n + 1}u(n)
    \label{eq:x-n-def}
\end{align}
\item Sketch 
\begin{align}
y(n)	 = x\brak{n-1} + x\brak{n+1},  \quad n \ge 0
\label{eq:10-orig-diff-rev}
\end{align}
\\
\solution
\begin{lstlisting}
$ https://github.com/sirichandra003/EE3900/blob/master/Assignments/Pingala%20Series/codes/q2.4.py
\end{lstlisting}
\begin{figure}[!htp]
    \includegraphics[width=\columnwidth]{figs/2.4.png}
    \caption{Plot of $x(n)$}
    \label{fig:xn}
\end{figure}
\item Find $Y^{+}(z)$. \\
\solution Taking the one-sided $Z$-transform on both sides of \eqref{eq:10-orig-diff-rev},
\begin{align}
&\mathcal{Z}^+\sbrak{y(n)} = \mathcal{Z}^+\sbrak{x(n + 1)} + \mathcal{Z}^+\sbrak{x(n - 1)} \\
&\sum_{n = 0}^{\infty}y(n)z^{-n}=\sum_{n = 0}^{\infty}x(n+1)z^{-n} + \sum_{n = 0}^{\infty}x(n-1)z^{-n} \\
&Y^+(z) = zX^+(z) - zx(0) + z^{-1}X^+(z) + zx(-1) \\
&=(z+z^{-1})X^+(z) - z \\
&= \frac{z + z^{-1}}{1 - z^{-1} - z^{-2}} - z \\
&= \frac{1 + 2z^{-1}}{1 - z^{-1} - z^{-2}}, \quad |z| > \alpha
\end{align}
since $x(n) = 0\ \forall\ n < 0$.
\item Find $y(n)$.		
\label{pr:1-3}
\\
\solution Using \eqref{eq:X-z},
\begin{align}
    Y^+(z) &= (1 + 2z^{-1})\sum_{n = 0}^{\infty}x(n)z^{-n} \\
           &= \sum_{n = 0}^{\infty}x(n)z^{-n} + \sum_{n = 1}^{\infty}2x(n - 1)z^{-n} \\
           &= x(0) + \sum_{n = 1}^{\infty}\brak{x(n) + 2x(n - 1)}z^{-n}
\end{align}
Given, $y(0) = x(0) = 1$ and equation $z^2 - z - 1 = 0$ has roots $\alpha$ and $\beta$ such that $\alpha + \beta = 1$ and $\alpha\beta =-1$,
\begin{align}
    y(n) &= \frac{\brak{\alpha^{n + 1} - \beta^{n + 1}} + \brak{2\alpha^n + 2\beta^n}}{\alpha - \beta} \\
         &= \frac{\brak{\alpha^{n + 2} - \beta^{n + 2}} + \brak{\alpha^{n} + \beta^{n}}}{\alpha - \beta} \label{eq:y-b} \\
         &= \frac{\brak{\alpha^{n + 2} - \beta^{n + 2}} - \alpha\beta\brak{\alpha^{n} + \beta^{n}}}{\alpha - \beta} \\
         &= \frac{\brak{\alpha - \beta}\brak{\alpha^{n + 1} + \beta^{n + 1}}}{\alpha - \beta} \\
         &= \alpha^{n + 1} + \beta^{n + 1}
\end{align}
$\therefore$ $y(n) = \alpha^{n + 1} + \beta^{n + 1}$ for $n \geq 0$ as $\alpha + \beta = 1$. Given $b_n = \alpha^n + \beta^n$. Thus by comparing \eqref{eq:y-b} $b_n$, we see that $y(n) = b_{n + 1}$.
\end{enumerate}
\section{Power of the Z transform}
\begin{enumerate}[label=\thesection.\arabic*,ref=\thesection.\theenumi]
\item Show that 
\begin{align}
	\sum_{k=1}^{n}a_k = 
	\sum_{k=0}^{n-1}x(n) = x(n)*u(n-1)
\end{align}
\label{pr:1-2}
 \\
\solution From \eqref{eq:x-n-def}, and noting that $x(n) = 0\ \forall\ n < 0$,
\begin{align}
    \sum_{k=1}^{n}a_k &= \sum_{k=0}^{n-1}x(k) \\
                      &= \sum_{k = -\infty}^{n - 1}x(k) \\
                      &= \sum_{k = -\infty}^{\infty}x(k)u(n - 1 - k) \\
                      &= x(n)*u(n - 1)
\end{align}
\item Show that 
\begin{align}
a_{n+2}-1, \quad n \ge 1
\end{align}
can be expressed as 
\begin{align}
	\sbrak{x\brak{n+1}-1}u\brak{n}
\end{align} \\
\solution From \eqref{eq:x-n-def},
\begin{align}
    a_{n+2} - 1 = \sbrak{x(n + 1) - 1}, \quad n \ge 0
\end{align}
and so, using the definition of $u(n)$,
\begin{align}
    a_{n+2} - 1 = \sbrak{x(n + 1) - 1}u(n)
\end{align}
 \item Show that 
\begin{align}
	\sum_{k=1}^{\infty}\frac{a_k}{10^k}= 
	\frac{1}{10}\sum_{k=0}^{\infty}\frac{x\brak{k}}{10^k} =\frac{1}{10}X^{+}\brak{{10}}
\end{align}
\solution 
\begin{align}
    \sum_{k=1}^{\infty}\frac{a_k}{10^k} &= \frac{1}{10}\sum_{k = 0}^{\infty}\frac{a_{k+1}}{10^k} \\
                                        &= \frac{1}{10}\sum_{k = 0}^{\infty}\frac{x(k)}{10^k} \\
                                        &= \frac{1}{10}X^+(10) \\
                                        &= \frac{1}{10}\times\frac{100}{89} = \frac{10}{89}
\end{align}
 \item Show that 
\begin{align}
	\alpha^n + \beta^n, \quad n \ge 1
	\label{eq:yn-exp}
\end{align}
can be expressed as 
\begin{align}
	w(n) =\brak{\alpha^{n+1} + \beta^{n+1}}u(n)
\end{align}
		and find $W(z)$.
\solution Putting $n = k + 1$ in \eqref{eq:yn-exp} and using the definition of $u(n)$, 
\begin{align}
\alpha^n + \beta^n = \brak{\alpha^{k + 1} + \beta^{k + 1}}u(k)
\end{align}
Hence, \eqref{eq:yn-exp} can be expressed as
\begin{align}
w(n) = \brak{\alpha^{n+1} + \beta^{n+1}}u(n) = y(n)
\end{align}
Therefore,
\begin{align}
    W(z) = Y(z) = \frac{1 + 2z^{-1}}{1 - z^{-1} - z^{-2}}
\end{align}
 \item Show that 
\begin{align}
	\sum_{k=1}^{\infty}\frac{b_k}{10^k} =
	\frac{1}{10}\sum_{k=0}^{\infty}\frac{y\brak{k}}{10^k} =\frac{1}{10}Y^{+}\brak{{10}}
\end{align}\label{pr:1-4}
\\
\solution
\begin{align}
    \sum_{k=1}^{\infty}\frac{b_k}{10^k} &= \frac{1}{10}\sum_{k = 0}^{\infty}\frac{b_{k+1}}{10^k} \\
                                        &= \frac{1}{10}\sum_{k = 0}^{\infty}\frac{y(k)}{10^k} \\
                                        &= \frac{1}{10}Y^+(10) \\
                                        &= \frac{1}{10}\times\frac{120}{89} = \frac{12}{89}
\end{align}
\item Solve the JEE 2019 problem.
\\
\solution We know that
\begin{align}
    \sum_{k = 1}^{n}a_k = x(n)*u(n - 1)
\end{align}
But
\begin{align}
    &x(n)*u(n - 1) \ztrans X(z)z^{-1}U(z) \\
    &= \frac{z^{-1}}{\brak{1 - z^{-1} - z^{-2}}\brak{1 - z^{-1}}} \\
    &= z\sbrak{\frac{1}{1 - z^{-1} - z^{-2}} - \frac{1}{1 - z^{-1}}} 
    &\ztrans z\sum_{n = 0}^{\infty}\brak{x(n) - 1}z^{-n} \\
    &= \sum_{n = 0}^{\infty}\brak{x(n) - 1}z^{-n + 1} \\
    &= \sum_{n = 0}^{\infty}\brak{x(n + 1) - 1}z^{-n} \\
\end{align}
From \eqref{eq:x-n-def}, we get
\begin{align}
    \sum_{k = 1}^{n}a_k = a_{n+2} - 1
\end{align}
We have already established the remaining options in order in the problems
\eqref{pr:1-2}, \eqref{pr:1-3}, \eqref{pr:1-4}. Therefore, options 1, 2,
and 3 are correct and option 4 is incorrect.
\end{enumerate}
\end{document}
